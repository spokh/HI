%%%%%%%%%%%%%%%%%%%%%%%%%%%%
\chapter{SUMMARY} \label{chap:6}
%%%%%%%%%%%%%%%%%%%%%%%%%%%%
Transport  properties of methane, and other light alkanes, ethane, propane, n-butane,  NO, CO , N2, in water have been studied with the  the help of calculating  self diffusion coefficient and mutual diffusion coefficients. Similarly, transport properties of alkali and halide ions in water have been studied by finding, mobility, diffusivity and friction coefficients of the constituents of the system in water. The structure of the system mentioned above have been  observed with the help of radial distribution functions and co-ordination numbers. The thermodynamic properties of the light alkanes 
(methane, ethane, propane and n-butane) in water and methanol has been studied by estimating the free energy of solvation of the alkanes in the respective solvent environment. All the calculations are based on  classical molecular dynamics simulations using empirical forcefields. 

 It can, therefore, be concluded that classical molecular dynamics simulation technique can be used as a reliable method to study the equilibrium structure and dynamic properties of fluid mixture. The calculated values of the diffusion coefficient may be used as a reference for any further fluid studies. In the near future, we want to study the diffusion coefficients by varying concentration of the component over wide range of temperature and also want to study the. 




 

