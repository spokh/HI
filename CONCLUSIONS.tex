%%%%%%%%%%%%%%%%%%%%%%%%%%%%%%%%%
\chapter{CONCLUSIONS AND RECOMMENDATIONS} \label{chap:5}

%%%%%%%%%%%%%%%%%%%%%%%%%%%%%%%%%
Transport  properties of methane, and other light alkanes, ethane, propane, n-butane,  NO, CO , N2, in water have been studied with the  the help of calculating  self diffusion coefficient and mutual diffusion coefficients. Similarly, transport properties of alkali and halide ions in water have been studied by finding, mobility, diffusivity and friction coefficients of the constituents of the system in water. The structure of the system mentioned above have been  observed with the help of radial distribution functions and co-ordination numbers. The thermodynamic properties of the light alkanes 
(methane, ethane, propane and n-butane) in water and methanol has been studied by estimating the free energy of solvation of the alkanes in the respective solvent environment. All the calculations are based on  classical molecular dynamics simulations using empirical forcefields.

 In this work, we have computed self diffusion coefficients along with binary coefficients of the system containing 971 water (H$_\mathrm{2}$O) molecules and 3 alkane (methane, ethane, propane, n-butane) molecules over a wide range of  temperatures from 283.15 K - 333.15 K,  using molecular dynamics simulation technique. The Extended Simple Point Charge (SPC/E)  model of water and Optimized Potential for Liquid Simulations-All Atom (OPLS-AA) of alkane were used. Here alkane molecule acts as a solute and water (H$_\mathrm{2}$O) as a solvent. Prior to the production run for the calculation of structure and transport properties, we monitored the temperature, energy, density of  the studied  system during equilibration to know the equilibrium state of the system. Structural properties has been studied using radial distribution function (RDF) and co-ordination number of the interaction cites has been calculated integrating RDF to the first co-ordination shell. The obtained RDFs  show that the system becomes less structured at high temperatures. The equilibrium structural properties of both the components (alkane and water) were studied calculating corresponding radial distribution function (RDF) namely $g_{OW-OW}(r)$ RDF of oxygen atoms of water molecules, $g_{CH_3-OW}(r)$ RDF of carbon atom of methyl group of alkane and oxygen atom of H$_\mathrm{2}$O , $g_{CH_2-OW}(r)$ RDF of carbon atom of methylene group of alkane and  oxygen atom of H$_\mathrm{2}$O.
 
  The main aim of our work was to study diffusion phenomenon of the mixture of water and alkane and study its temperature dependence. The self-diffusion coefficients of water and alkane (methane, ethane, propane and n-butane) was estimated using Einstein's method  separately. The diffusion coefficients of water are deviated within 11$\%$ of the available experimental data. The binary  diffusion coefficient of the system was calculated using Darken's relation. The values of binary diffusion coefficients of alkane in water do not agree well with the  previous  experiments. values . It lies in between these two experimental values of D. L. Wise and G. Houghton and  P. A. Witherspoon and D. N.
Saraf and the deviation is increasing with increase in temperature. The Arrhenius diagram (plot of natural logarithm of diffusion coefficient vs inverse of temperature) was plotted for self-diffusion coefficients of water and binary diffusion of the alkane-water system  separately and it showed temperature dependence of diffusion coefficient of both are of Arrhenius type.

Again, we took the binary mixture of the system containing 280 water (H$_\mathrm{2}$O) molecules and 5 nitric oxide (NO) molecules,  280 water (H$_\mathrm{2}$O) molecules and 5  carbon-monoxide and 265 water (H$_\mathrm{2}$O) molecules and 5 nitrogen molecules, each of separate system and   diffusion
of the gaseous systems  in water for various temperatures from 293 K to 333 K   has been estimated using molecular dynamics simulation technique. For simulation procedure GROMACS 4.6.5 software was used. The input parameters were taken so as to be consistent with the experimental values as much as possible. The Extended Simple Point Charge (SPC/E)  model of water was used. Here the gases (NO, CO  and N2)  act as  solutes and water (H$_\mathrm{2}$O) as a solvent. Our working procedure could be divided into three important parts, energy minimization, equilibration and production run. Steepest descent method was used for energy minimization process. To equilibrate, the system was simulated for 200 ns in NPT ensemble using Berendsen thermostat and barostat. The system was equilibrated to the desired temperature and a pressure of 1 atm ($1.01\times10^5\; Pa$). The system was then operated for production run using NVT ensemble for 200 ns with the time step of 2 fs. The energy profile of the system were studied to know the equilibrium nature of the system. Structural properties has been studied using Radial Distribution Function (RDF), which shows the system becomes less structured at high temperatures. 

The equilibrium structural properties of both the components (NO, and H$_\mathrm{2}$O) were studied calculating corresponding radial distribution function (RDF) namely $g_{OW-OW}(r)$ RDF of oxygen atoms of water molecules, $g_{ON-OW}(r)$ RDF of oxygen atom of NO and oxygen atom of H$_\mathrm{2}$O, $g_{ON-ON}(r)$ RDF of oxygen atoms of NO molecules. Similarly, for CO in water, the equilibrium structural properties of both the components (CO and $H_2 O$) were studied calculating corresponding radial distribution function (RDF) namely $g_{ow-ow}(r)$ RDF of oxygen atoms of water molecules, $g_{oc-ow}(r)$ RDF of oxygen atom of CO and oxygen atom of $H_2 O$, $g_{oc-oc}(r)$ RDF of oxygen atoms of CO molecules. Furthermore, the equilibrium structural properties of both the components, nitrogen and water, were studied by calculating the corresponding partial radial distribution functions, namely $g_{N-N}(r)$, $g_{N-W}(r)$, $g_{O-O}(r)$  without employing any long range corrections. The RDF of $g_{ON-ON}(r)$, $g_{oc-oc}(r)$, $g_{N-N}(r)$  showed some convolutions and depression at short distance due to less number of the gaseous molecules.

 The main aim of our work was to study diffusion phenomenon of the mixture of water and gaseous molecules and study its temperature dependence. The self-diffusion coefficients of water was estimated using Einstein's method and that of NO was estimated using Einstein's and Green-Cubo relation (VACF method) separately. The values of diffusion coefficients of NO at low temperature agree well with the experimental results~\citep{zacharia2005diffusivity}. But, the  values of diffusion coefficient of NO up to 313K is deviated within 32.94$\%$ of the available experimental data~\citep{wise1968diffusion}, and that at higher temperature reaches up to 65.7 $\%$ at 333K. The diffusion coefficients of water are deviated within 9.22$\%$ of the available experimental data~\citep{easteal1989diaphragm}. The values of the self-diffusion coefficient for NO obtained using MSD and VACF varies within 5$\%$.  The values obtained by these two methods for nitrogen molecules are in excellent agreement. Also they are in agreement with the available experimental values~\citep{verhallen1984diffusion, ferrell1967diffusion} with  maximum deviation of about 14\%.  The simulated values of the
 self-diffusion coefficient of CO using MSD method varies within 2\% maximum
 in comparison to VACF method. The self-diffusion coefficient estimated using
 both the methods varies within 15\% to that of experimental values~\citep{wise1968diffusion} except
 at temperatures T=313 K and T=323 K, and that of water is deviated within
 13\% of the available experimental data~\citep{easteal1989diaphragm,mills1973self}, which highlights the goodness  of SPC/E water model. The binary or mutual diffusion coefficient of the system
 was calculated using Darken's relation. The Arrhenius diagram (plot of natural
 logarithm of diffusion coefficient versus inverse of temperature) was plotted for
 self-diffusion coefficient of CO, NO, N2 and H2O separately and it showed temperature dependence of self-diffusion coefficient of both is of Arrhenius type. 
  
 Similarly, to study the diffusivity and  mobility of ions, we took the system containing 305 water (H$_\mathrm{2}$O) molecules, 5 ions   and 5 counter ions for making the system neutral. The TIP3P  model of water was used and the OPLSS-AA  potential model for the ions. The equilibrium  structural properties has been studied using Radial  Distribution Function (RDF) and co-ordination numbers (ion-hydration number) of the interaction cites.  The diffusion coefficients, mobility and friction coefficients of ions were estimated  using Einstein's and velocity autocorrelation method separately and diffusion coefficient of  water was estimated using Einstein's method only. The values of diffusion coefficients, mobility and friction coefficients  of different ions  agree well with the experimental
 results~\citep{lide2009crc}. The values of the diffusion coefficient for ions obtained using MSD and VACF are within maximum deviation of  8$\%$ for $Br^-$). The  simulations of ion mobility in TIP3P water show the same trends with respect to size and charge type observed in  experiments and also provide information on the solvation structure and dynamics that bear indirectly on ion mobilities in
 aqueous solutions. It  provide  directions for future theoretical developments of diffusion and  mobility  of ions in the aqueous solutions. 

Also, interactions of simple alkanes (methane, ethane, propane  and n-butane) in different solvent environments can be a model research for the hydrophobic interactions  in biological molecules. The solvent environment around the hydrophobic groups affect the strength and nature of interactions which may reveal the conditions for protein folding and their denaturation. We consider water and  methanol,  as polar and amphiphilic  solvents respectively to study their effect on calculating free energy of solvation of the light and simplest alkanes.   

 It can, therefore, be concluded that classical molecular dynamics simulation technique can be used as a reliable method to study the equilibrium structure and dynamic properties of fluid mixture. The calculated values of the diffusion coefficient may be used as a reference for any further fluid studies. In the near future, we want to study the diffusion coefficients by varying concentration of the component over wide range of temperature and also want to study the.  
 
 It is well-known that liquid water form their network around the hydrophobic groups. The microscopic understanding of their  probability forming networks of different size and  nature helps to predict many biological and chemical processes. Previous studies   have used ring statistics (formation of water rings) around the hydrophobic molecules for such purpose~\citep{Hassanali2013}.  There are many directions we can take in the future. One of them would be to find a good  method to make entropy-enthalpy decomposition for solvation of the alkanes in water and methanol, just to  see the system in a better light. Also we saw from the discussion on the hydrophobic effect that  it is a vast subject and we can try to see what implications does it have on systems other than  water. Even the work where we would solvate water in a non-polar environment could lead to  some new findings. Moreover,  we can perform MD study of large macromolecules, proteins,  nucleic acids and  lipids and  calculate the free energy for molecular solids for exploring the phase diagram of solid methane.


 






