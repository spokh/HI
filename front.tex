%%%%%%%%%%%%% title outer
\begin{titlepage}
    \centering
        {\large\bf TRANSPORT AND THERMODYNAMIC PROPERTIES \\OF METHANE AND VARIOUS GASES IN WATER}
    
    \vspace{2cm}
    
    \begin{figure}[!h]
    \centering
    \includegraphics[width = 4.3cm, keepaspectratio]{logo.png}
    \end{figure}
    
    \vspace{1cm}
    
    \begin{center}
    { A THESIS SUBMITTED TO THE \\
    \textbf{CENTRAL DEPARTMENT OF PHYSICS \\
    INSTITUTE OF SCIENCE AND TECHNOLOGY \\
    TRIBHUVAN UNIVERSITY \\
    NEPAL}}
    
    \vspace{2cm}
    
    {\bf FOR THE AWARD OF \\
    DOCTOR OF PHILOSOPHY \\
    IN PHYSICS}
    \end{center}
    
    \vspace{2cm}
    
    \begin{center}
    {\large By} \\
    {\textbf{SUNIL POKHAREL}}	
    
    \vfill
    
    %{\date:\today}
    {\ \bf AUGUST 2018}
    \end{center}
\end{titlepage}

\mbox{}
\thispagestyle{empty}
\newpage
%%%%%%%%%%%%% title inner
\begin{titlepage}
     \centering
    {\large\bf TRANSPORT AND THERMODYNAMIC PROPERTIES \\OF METHANE AND VARIOUS GASES IN WATER}
    
    \vspace{2cm}
    
    \begin{figure}[!h]
    \centering
    \includegraphics[width = 4cm, keepaspectratio]{logo.png}
    \end{figure}
    
    \vspace{1cm}
    
    \begin{center}
    { A THESIS SUBMITTED TO THE \\
    \textbf{CENTRAL DEPARTMENT OF PHYSICS \\
    INSTITUTE OF SCIENCE AND TECHNOLOGY \\
    TRIBHUVAN UNIVERSITY \\
    NEPAL}}
    
    \vspace{2cm}
    
    {\bf FOR THE AWARD OF \\
    DOCTOR OF PHILOSOPHY \\
    IN PHYSICS}
    \end{center}
    
    \vspace{2cm}
    
    \begin{center}
    {\large By} \\
    {\textbf{SUNIL POKHAREL}}	
    
    \vfill
    
    %{\date:\today}
    {\ \bf AUGUST 2018}
    \end{center}
\end{titlepage}
%%%%%%%%%%%%% declaration
\newpage
\phantomsection
\addcontentsline{toc}{chapter}{Declaration}
\begin{center}
\Large\bf DECLARATION 
\end{center}
\vspace*{2.0cm}
 
Thesis entitled {\bf \lq\lq Transport and thermodynamic properties of methane and various gases in water  \rq \rq} which is being submitted to the Central Department of Physics, Institute of Science and Technology (IOST), Tribhuvan University,
Nepal, for the award of the degree of Doctor of Philosophy (Ph.D.), is a research work carried out by me under the supervision of Prof. Dr. Narayan Prasad Adhikari, Central Department of Physics, Tribhuvan University, Kirtipur, Nepal. 

This research is original and has not been submitted earlier in part or full in this or any other form to any university or institute, here or elsewhere, for the award of any degree.

\vskip 4.0cm                                                                                               

\hfill Sunil Pokharel \\
%%%%%%%%%%%%% Recommendation
\newpage
\phantomsection
\addcontentsline{toc}{chapter}{Recommendation}
\begin{center}
\Large\bf RECOMMENDATION
\end{center}
\vspace*{1.0cm}

This is to recommend that {\bf Sunil Pokharel} has carried out research entitled {\bf \lq \lq Transport and thermodynamic properties of methane  and various gases in water \rq \rq } for the award of Doctor of Philosophy (Ph.D.) in {\bf Physics} under my supervision. To my knowledge, this work has not been submitted for any other degree. 

He has fulfilled all the requirements laid down by the Institute of Science and Technology (IOST), Tribhuvan University, Kirtipur for the submission of the thesis for the award of Ph.D. degree.

\vspace{0.5cm}
\dots\dots\dots\dots\dots\dots\dots\dots\dots \\
\textbf{ Dr. Narayan Prasad Adhikari}, \\
\textbf{Professor and Supervisor} \\
Central Department of Physics,\\
Tribhuvan University,\\
Kirtipur, Kathmandu, \\
Nepal


\vspace{5cm}
\begin{center}
{\bf AUGUST 2018}
\end{center}
%%%%%%%%%%%%% Approval
\newpage
\phantomsection
\addcontentsline{toc}{chapter}{Letter of Approval}
\centerline{\Large\bf LETTER OF APPROVAL}

\vspace*{2.0cm}

[Date:08/10/2018] 

On the recommendation of \textbf{Prof. Dr. Narayan Prasad Adhikari} of Central Department of Physics, Tribhuvan, University, Kirtipur, Kathmandu, Nepal, this Ph.D. thesis submitted by \textbf{Sunil Pokharel}, entitled ``\textbf{Transport and thermodynamic properties of methane  and various gases in water }'' is forwarded by Central Department Research Committee (CDRC) to the Dean, Institute of Science and Technology (IOST), Tribuvuan University (TU).

\vskip 3.0cm
\dots\dots\dots\dots\dots\dots\dots\dots\dots \\
{\bf Dr. Binil Aryal} \\
Professor and Head \\
Central Department of Physics, \\
Tribhuvan University, \\
Kirtipur, Kathmandu, \\
Nepal

%%%%%%%%%%%%% Acknowledgement
\newpage
\phantomsection
\addcontentsline{toc}{chapter}{Acknowledgements}
\centerline{\Large\bf ACKNOWLEDGMENTS}
\vspace{1cm}
Undertaking this Ph.D. has been  a  truly life-changing experience for me  and it would not have been possible  to do without the support and guidance  that I received from many people and institutions.   

First and foremost, I would like to express my sincere gratitude to my supervisor Prof. Dr. Narayan Prasad Adhikari for the continuous support of my Ph.D. study and research, for his patience, motivation, enthusiasm, and immense knowledge. I would like to thank you for encouraging my research and for allowing me to grow as a research scientist. I appreciate all your contributions of time, ideas, and funding to make my Ph.D. experience productive and stimulating. Your advice on both research as well as on my career have been priceless.

I am extremely grateful to Prof. Dr. Binil Aryal, Head of the Central Department of
Physics, Prof. Dr. Devendra Raj Mishra, Prof. Dr. Uday Raj Khanal, Prof. Dr. Jeevan
Jyoti Nakarmi,  Prof. Dr. Harihar Paudyal, and Prof. Dr. Raju Khanal for their supports and encouragements during my Ph. D. studies. 




Next, I gratefully acknowledge the receipt of the grant from the Abdus Salam International Centre for Theoretical Physics (ICTP), Trieste, Italy through the Office of External Activities (OEA) for the PhD studies. I also appreciate TWAS (The World of Academy for Sciences) for the computing facilities.

I am also thankful to my seniors Dr. Gopi Chandra Kaphle (TU), and Dr. Nurapati Pantha (TU) for their moral and academic support. I always found Dr. Kaphle and Dr. Pantha as my mentors and guardians  who take care of my sentiments and emotions during my difficult time, in addition to valuable scientific instructions. My sincere thanks goes to my labmates Saran Lamichhane, Shyam Prakash Khanal, Rajendra Prasad Koirala, Ishwar Poudyal, Dipendra Bhandari,  Narayan Aryal, Arjun Subedi, Rabindra Oliya, Bhakta Raj Niraula  and more than a dozen of master's thesis students at Condensed Matter and Materials Sciences group, Central Department of Physics, to whom I discussed about science and daily life. I would like to express my thanks to all the faculty members and and the staffs of the Central Department of Physics (CDP), Department of Physics, Patan Multiple Campus and Department of Physics, Amrit Campus, who helped me for varying nature of works without which my thesis work would not be of present shape.


 I would like to thank my family for all their love and encouragement. Words cannot express how grateful I am to my mother, mother-in-law, sisters and brothers for all of the sacrifices that they've made on my behalf. Their prayer for me was what sustained me thus far. I would also like to thank all of my friends, especially Bashu, Madan, Ramesh, Devendra, Hari, Yogesh who supported me in every stages of the Ph. D. work, and incented me to strive towards my goal. At the end I would like to express appreciation to my loving, supportive, encouraging, patient, and   beloved wife Pushpa Pandey whose faithful support during the final stages of this Ph.D. is so appreciated.

.\vskip 1.0in

\begin{flushright}
$\dots\dots\dots\dots\dots\dots\dots $ \\{\text{Sunil Pokharel}}\\
June, 2018 
\end{flushright}
%%%%%%%%%%%%% Abstract
\newpage
\phantomsection
\addcontentsline{toc}{chapter}{Abstract}
\begin{center}
\Large\bf{ABSTRACT}
\end{center}

\vspace{0.25cm}
{\small{Fast improvement of technology in the recent decades contributed to computer simulations becoming
a major phase of scientific research, developing parallel with theoretical and experimental methods. Computer simulations of molecular models are powerful technique that have improved the understanding of many biochemical phenomenon. The method is frequently applied to study the motions of simple atomic systems and complex  biological macromolecules such as proteins and lipids. Molecular Dynamics simulations  provide insight into the microscopic behavior of systems that we cannot see in the experiments. Diffusion, transport of mass in response to concentration and thermal energy gradient, is an important transport property, vital in material science and life science. Methane related compounds are frontier topics in research and industrial sectors because of their relevancy in energy to environment, geophysics to astronomy and physical chemistry to biological sciences. The diffusion, solubility or hydrophobicity of hydrocarbons (alkanes) in different  solvent environments is a basic consideration in many processes like processing of natural gases and petroleum, understanding hydrophobic interactions and phenomenon of protein folding. In this work, we have used atomistic level of calculations to estimate diffusion coefficient of methane and various gases (ethane, propane, n-butane, CO, NO, and  N2) in water and diffusivity and mobility some alkali  and halide ions in water. In addition to this,  free energy of solvation of  light alkanes  in different liquid-media (water and methanol)  was calculated by means of thermodynamic integration.

In the present work, Molecular Dynamics (MD) simulation of dilute system of light alkanes (methane, ethane , propane and  n-butane) in water (3 alakne in 971 water), each of a separate system, are performed at ten different temperatures  from 283.15 K to 333.15 K  under a constant pressure (P=1.013 bar)  using GROMACS-Groningen Machine for Chemical Simulations. Force field parameters are used according to Optimized Potentials for Liquid Simulations-All Atom for alkanes and SPC/E for water. The structural property of the system is studied through different pairs of the atoms in the system which shows more than one peaks suggesting that there exists noticeable interactions between the atoms in the system. Self-diffusion coefficients of the alkane and water are estimated by means of Mean Square Displacement (MSD) plot exploiting Einstein's relation. Mutual diffusion coefficient of the system
is calculated by using Darken's relation. Values of self-diffusion coefficient of water
estimated in this present work are in excellent agreement with the experimental data
available in the literature. Though the diffusion coefficients of binary mixture at
low temperatures are close to experimental values, the values at higher temperatures
are deviated noticeably from the experimental values with the maximum deviation
of 39.53\% at 333.15 K for n-butane. 

Again,  Molecular Dynamics simulations of a mixture of various gases (CO, NO and N2)
 in SPC/E water (H 2 O) with the gases as solutes  and water as a solvent, each of different system have been performed to understand the self-diffusion coefficient of the components i.e. the gases and water along with the mutual diffusion coefficients at different temperatures from 293 K to 333 K at a costant pressure of 1 bar. The mole fraction of the solute ( CO, NO, and N2) in the
system is  0.018. The solvent-solvent, solute-solute
and solute-solvent radial distribution functions have been estimated to study
the structural properties of the system. The self-diffusion coefficient of the gases (CO, NO and N2) is
calculated using mean square displacement (MSD) and velocity auto correlation
function (VACF) and that of water is estimated using MSD method only. The temperature dependence of self-diffusion coefficients of the gases  and water and mutual diffusion coefficients of the gases in water all follow the Arrhenius behavior from where we estimate the activation energy of diffusion process. Activation energies are estimated using Arrhenius diagrams and
these values are found to be in good agreement with the experimental values. In addition to this, we have estimated  the diffusion coefficients  and mobilities of alkali metal cations $Na^{+}$, $K^{+}$ and halide anions $F^{-}$, $Cl^{-}$, $Br^{-}$, $I^{-}$  at infinite dilution in water at 25 $^{\circ}$C. The diffusivity and the mobility of ions are  estimated  by using mean square displacement (MSD) and  velocity auto correlation function (VACF) methods. The results are then compared with the available experimental values. The ion diffusivity and  mobilities show the same trends as the experimental
 results with distinct maxima for cations and anions. The role of interaction in the structure of the constituents of the system is studied with the help of the radial distribution function (RDF) and the coordination numbers. 
 
At the end, we have estimated the solvation free  energy for light alkane (methane, ethane, propane and n-butane) dissolved in water and methanol  respectively over a broad range of temperatures, from 275 K to 375 K, using molecular dynamics simulations. The alkane (methane, ethane, propane and n-butane), and methanol molecules are  described by the OPLS-AA  potential, while water is modeled by TIP3P  model. We  have used the free energy perturbation method (Bennett Acceptance Ratio (BAR) method) for the  calculation of free energy of solvation. The estimated values of solvation free energy of alkane in  the correspnding solvents agree well with the available experimental data. 


The present research provides a powerful molecular dynamics simulation-based framework that will enable the development of more complex models of study of small molecules and  large macromolecules.}}

%%%%%%%%%%%%% Abbreviations
\newpage
\phantomsection
\addcontentsline{toc}{chapter}{List of Acronyms and Abbreviations}
\begin{center}
{\Large\bf LIST OF ACRONYMS AND ABBREVIATIONS}
\end{center}
\vspace{1cm}
\begin{tabular}{l p{10cm}}

AMBER    & Assisted Model Building with Energy Refinement\\
BAR      & Bennett Acceptance Ratio\\
B. E.    & Binding Energy \\
BOA      & Born-Oppenheimer Approximation \\
CE       & Chapman-Enskog\\
CHARMM   & Chemistry at Harvard Macromolecular Mechanics\\
CNG      & Condensed Natural Gases \\
CNT      & Carbon Nanotube \\
CVD      & Chemical Vapor Deposition \\
DFT      & Density Functional Theory \\
DLS      & Dynamic Light Scattering \\
DOS      & Density of States \\
ER       & Excluded Region \\
ESR      & Electron-Spin Resonance\\
eV       & Electron Volt \\
FPP      & First Peak Position\\
FPV      & Fist Peak Value \\
fs       & femtosecond\\
GPa      & Giga Pascal \\
GROMACS  & GROningen MAchine for Chemical Simulations \\
GROMOS   & GROningen Molecular Simulation \\
ICTP    & Abdul Salam International Center For Theoretical Physics \\
 K      & Kelvin \\
K. E.   & Kinetic Energy \\
L-BFGS  & limited-memory-Broyden-Fletcher-Goldfarb-Shanno quasi-Newtonian-mimimizer \\
LJ      & Lennard Jones\\ 
LNG     & Liquefied Natural Gases \\

\end{tabular}
\newpage
\begin{tabular}{l p{10cm}}
MD      & Molecular Dynamics \\
MC      & Monte Carlo \\
MSD     & Mean Square Displacement\\
NMR     & Nuclear Magnetic Resonance \\
nm      & nanometer\\
ns      & nanosecond\\
OPLS-AA & Optimized Potentials for Liquid Simulations-All Atom \\
PBC     & Periodic Boundary Condition \\
ps      & picosecond\\
RDF     & Radial Distribution Function\\
Ry      & Rydberg \\
SDS     & sodium dodecyl sulfate \\
SPC     & Simple Point Charge\\
SPC/E   & Simple Point Charge-Extended\\
SPP     & Second Peak Position \\
SPV     & Second Peak Value \\
STM     & Scanning Tunneling Microscopy \\
SWCNTs  & Single-Walled carbon NanoTubes \\
TI      & Thermodynmic Integration\\
TIP3P   & Transferable IntermolecularPotential with 3-Points \\
TIP4P   & Transferable IntermolecularPotential with 4-Points\\
TIP5P   & Transferable IntermolecularPotential with 5-Points\\
TraPPE  & Transferable Potentials for Phase Equilibria  \\
TU      & Tribhuvan University \\
TWAS    & Third World Academy of Science \\
vdW     & van der Waals \\
VACF    & velocity autocorrelation function\\
VMD     &  Visual Molecular Dynamics\\
XRD     & X-ray Diffraction
\end{tabular}